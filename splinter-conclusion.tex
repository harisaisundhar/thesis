\section{Conclusion}
\label{sec:conclusion}

Our work shows that, by combining ideas from LSM trees and \bets, we can build
a \kv store that outperforms current \kv stores by up to an order of magnitude
on insertions, matches or outperforms on lookups, and is competitive on range
queries.

\Sysname targets the common case of small key-value pairs and
non-uniformly random workloads.  Many real-world \kv workloads come
from different clients, some of which might be performing very
localized operations, while others are performing relatively random
operations.  \Sysname exploits whatever locality is available.

\Sysname makes contributions to both the data-structural and systems
design of high-performance \kv stores.  We show how to get the low
write amplification of size-tiered data structure while maintaining
the high query throughput and workload-adaptivity of a \bet.  We also
describe several systems issues, such as cache, lock, and memtable
design, that one must address to extract the full performance of
high-performance NVMe devices.
