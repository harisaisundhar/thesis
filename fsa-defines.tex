% comment out the following line when submitting
% \def \ftfsdebug{}
%\def \ftfsreviews{}

% Make list indentation tight\
\setdefaultleftmargin{1em}{0em}{}{}{}{}
%\setlength{\droptitle}{-.5in}     % Eliminate the default vertical space
\captionsetup{belowskip=0pt,aboveskip=4pt}
\newcolumntype{d}[1]{D{.}{.}{#1}}

% Figure adjustments for space
\renewcommand{\topfraction}{0.95}
\renewcommand{\dbltopfraction}{0.95}
\setcounter{dbltopnumber}{4}
\renewcommand\textfraction{.05}
\setcounter{totalnumber}{5}
\renewcommand{\floatpagefraction}{.95}
\renewcommand{\textfloatsep}{8pt}
\renewcommand{\dbltextfloatsep}{8pt}

\newcommand{\tightpara}[1]{\vspace{5pt} \noindent \textbf{#1}}

%% draw a circled around some text. can be used for labels
\newcommand*\circled[1]{\tikz[baseline=(char.base)]{
    \node[shape=circle,draw,inner sep=0.5pt] (char) {#1};}}

\definecolor{cerisepink}{rgb}{0.93, 0.23, 0.51}
\definecolor{ballblue}{rgb}{0.13, 0.67, 0.8}

\ifx \ftfsdebug \undefined
\newcommand{\fixme}[1]{{}}
\newcommand{\fixmedp}[1]{{}}
\newcommand{\fixmeac}[1]{{}}
\newcommand{\fixmemfc}[1]{{}}
\newcommand{\fixmerob}[1]{{}}
\newcommand{\fixmemab}[1]{{}}
\else
\newcommand{\fixme}[1]{{\bf\textcolor{red}{ [ FIXME: #1 ]}}}
\newcommand{\fixmedp}[1]{{\bf\textcolor{red}{ [ FIXME dP: #1 ]}}}
\newcommand{\fixmeac}[1]{{\bf\textcolor{red}{ [ FIXME AC: #1 ]}}}
\newcommand{\fixmemfc}[1]{{\bf\textcolor{red}{ [ FIXME MFC: #1 ]}}}
\newcommand{\fixmerob}[1]{{\bf\textcolor{red}{ [ FIXME ROB: #1 ]}}}
\newcommand{\fixmemab}[1]{{\bf\textcolor{red}{ [ FIXME MAB: #1 ]}}}
\newcommand{\shepherd}[1]{{\bf\textcolor{YellowOrange}{ [ PHIL: #1 ]}}}
\fi
\newcommand{\mfc}[1]{{\fixmemfc{#1}}}

\newcommand{\del}[1]{{~\color{orange} #1 \marginpar{To Delete}}}



\ifx \ftfsreviews \undefined
\newcommand{\reviewer}[1]{{}}
\else
\newcommand{\reviewer}[1]{{\bf\textcolor{red}{ [ REVIEWER: #1 ]}}}
\fi

\ifx \ftfsreviews \undefined
\newcommand{\reviewerfixed}[2]{{#2}}
\newcommand{\reviewerfixedyang}[2]{{#2}}
\else
\newcommand{\reviewerfixed}[2]{{\bf\textcolor{cerisepink}{ [ REVIEWER(#1): #2]}}}
\newcommand{\reviewerfixedyang}[2]{{\bf\textcolor{ballblue}{ [ REVIEWER(#1): #2]}}}
\fi



\newcommand{\punt}[1]{}
\newcommand{\wod}{WOD\xspace}
\newcommand{\wods}{WODs\xspace}
\newcommand{\wofs}{write-optimized file system\xspace}
\newcommand{\wofss}{write-optimized file systems\xspace}

\newcommand{\ft}{Fractal Tree\xspace}
\newcommand{\fti}{\ft index\xspace}
\newcommand{\FTI}{FTI\xspace}
\newcommand{\FTIs}{FTIs\xspace}
\newcommand{\ftis}{\ft indexes\xspace}
\newcommand{\ftistitle}{\ft Indexes\xspace}
\newcommand{\lsm}{LSM-tree\xspace}
\newcommand{\lsms}{LSM-trees\xspace}
\newcommand{\vttree}{VT-tree\xspace}
\newcommand{\vttrees}{VT-trees\xspace}


\newcommand{\kvbfs}{key-value-based file system\xspace}
\newcommand{\kvbfses}{\kvbfs{}s\xspace}
\newcommand{\Kvbfs}{Key-value-based file system\xspace}
\newcommand{\Kvbfses}{\Kvbfs{}s\xspace}
\newcommand{\bet}{B$^{\varepsilon}$-tree\xspace}
\newcommand{\beT}{B$^{\varepsilon}$-Tree\xspace}
\newcommand{\bets}{B$^{\varepsilon}$-trees\xspace}
\newcommand{\sysname}{BetrFS\xspace}
\newcommand{\betrfs}{BetrFS\xspace}
\newcommand{\Ext}{{Ext4}\xspace}
\newcommand{\ext}{{ext4}\xspace}
\newcommand{\btrfs}{{Btrfs}\xspace}
\newcommand{\Btrfs}{{Btrfs}\xspace}
\newcommand{\xfs}{{XFS}\xspace}
\newcommand{\zfs}{{ZFS}\xspace}
\newcommand{\Zfs}{{Zfs}\xspace}
\newcommand{\ftwofs}{{F2FS}\xspace}
\newcommand{\tablefs}{{TableFS}\xspace}
\newcommand{\kvfs}{{KVFS}\xspace}
\newcommand{\tokufs}{{TokuFS}\xspace}

\newcommand{\libc}{{\tt libc}\xspace}
\newcommand{\klibc}{{\tt klibc}\xspace}
\newcommand{\grep}{\texttt{grep}\xspace}
\newcommand{\blktrace}{\texttt{blktrace}\xspace}
\newcommand{\defn}[1]           {\textit{\textbf{\boldmath #1}}}
\newcommand{\proc}[1]		{\ifmmode\mbox{\textsc{#1}}\else\textsc{#1}\fi}

\newcommand{\gb}{{GB}\xspace}
\newcommand{\gib}{{GiB}\xspace}
\newcommand{\mb}{{MB}\xspace}
\newcommand{\mib}{{MiB}\xspace}
\newcommand{\kib}{{KiB}\xspace}
\newcommand{\mbps}{{\mb/s}\xspace}
\newcommand{\mibps}{{\mib/s}\xspace}
% standardize benchmarking operations, libc text references
\newcommand{\lmdd}{\texttt{lmdd}\xspace}
\newcommand{\rsync}{\texttt{rsync}\xspace}
\newcommand{\fsync}{\texttt{fsync}\xspace}
\newcommand{\mkdir}{\texttt{mkdir}\xspace}
\newcommand{\unlink}{\texttt{unlink}\xspace}
\newcommand{\truncate}{\texttt{truncate}\xspace}
\newcommand{\memcpy}{\texttt{memcpy}\xspace}
\newcommand{\pread}{\texttt{pread}\xspace}
\newcommand{\pwrite}{\texttt{pwrite}\xspace}
\newcommand{\libcread}{\texttt{read}\xspace}
\newcommand{\libcwrite}{\texttt{write}\xspace}

\newcommand{\linuxver}{{$3.11.10$}\xspace}

% It makes no sense to use cref for typed references such as secref.
% You lose the type checking.  For example if I
% write \secput{intro}{Introduction} and later write \figref{intro},
% this code will give undefined warnings.  If I write \cref, the
% system will automagically turn that into a Section~\ref{secintro}
% which won't be what you wanted, since I was trying to reference a
% figure.
%
% It makes doubly no sense to explicitly write \cref{fig:foo} instead
% of \figref{foo}.
%
% -Bradley
\newcommand{\secput}[2]{\section{#2}\label{sec:#1}}
\newcommand{\secref}[1]{Section~\ref{sec:#1}}     
\newcommand{\subsecput}[2]{\subsection{#2}\label{sec:#1}}
\newcommand{\figputt}[2]{%
 \begin{figure}[t]%
 \begin{center}%
  \input{#1.pdf_tex}%
 \end{center}%
 \caption{#2}%
 \label{fig:#1}%
 \end{figure}%
}
\newcommand{\figlabel}[1]{\label{fig:#1}}
\newcommand{\figref}[1]{Figure~\ref{fig:#1}}

\newcommand{\DATA}[1]{\textcolor{red}{#1}}
