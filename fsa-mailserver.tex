\section{Application Level Aging: Mail Server}\label{sec:fsa-mailserver} In
addition to the git workload, we evaluate aging with the Dovecot email server.
Dovecot is configured with the Maildir backend, which stores each message in a
file, and each inbox in a directory.  We simulate 2 users, each having 80
mailboxes, receiving new email, deleting old emails, and searching through their
mailboxes. 

A cycle or ``day'' for the mailserver comprises 8,000 operations, where each
operation is equally likely to be an insert or a delete, corresponding to
receiving a new email or deleting an old one. Each email is a string of random
characters, the length of which is uniformly distributed over the range [1,
32K]. Each mailbox is initialized with 1,000 messages, and, because inserts and
deletes are balanced, mailbox size tends to stay around 1,000.  We simulate the
mailserver for 100 cycles and after each cycle we perform a recursive grep for
a random string. As in our git benchmarks, we then copy the partition to a
freshly formatted file system, and run a recursive grep.

\cref{fig:mailserver} shows the read costs in seconds per GiB of the grep test
on hard disk.  Although the unaged versions of all file systems show consistent
performance over the life of the benchmark, the aged versions of \ext, \btrfs,
\xfs and \zfs all show significant degradation over time. In particular, aged
\ext performance degrades by 4.4$\times$, and is 28$\times$ slower than aged
\betrfs.  \xfs slows down by a factor of 7 and \btrfs, by a factor of 30. \zfs
slows down drastically, taking about 20 minutes per GiB by cycle 20.  However,
the aged version of \betrfs does not slow down. As with the other HDD
experiments, dynamic layout score is moderately correlated ($-0.63$) with grep
cost.

%
% File parsing definitions
%
\newcommand{\mailserverDataDir}{fsa-data/mailserver_aging}
% Args: hardware, partition-size, file-system
\newcommand{\mailserverDataFileName}[3]{\mailserverDataDir/mailserver_#3_#1_#2.csv}
\newcommand{\mailserverOperationCountColumn}{pulls_performed}
\newcommand{\mailserverFSSizeColumn}{filesystem_size}
\newcommand{\mailserverLayout}[1]{#1_layout_score}

% construct a table with the reference FS sizes from ext4 on hdd
\pgfplotstableread{\mailserverDataFileName{hdd}{20gb}{ext4}}\extfourmsdata

% args are: FS agedness hardware partition-size
\newcommand{\addmailserverplot}[4]
{
  \pgfplotstableread{\mailserverDataFileName{#3}{#4}{#1}}\thistable
  \pgfplotstablecreatecol[copy column from table={\extfourmsdata}{\mailserverFSSizeColumn}]{ext4_fs_size}{\thistable}
  \addplot[color=\pgfkeysvalueof{/fs-colors/#1}, style=\pgfkeysvalueof{/agedness-styles/#2}, line width=0.75pt, mark=\pgfkeysvalueof{/fs-marks/#1}, mark repeat=5] 
  table[x=\mailserverOperationCountColumn, y expr=\thisrow{\pgfkeysvalueof{/agedness-columns/#2}} / \thisrow{ext4_fs_size} * 1048576] \thistable;
  %\addfsplot{#1}{#2}{\pullCountColumn}{\thisrow{\pgfkeysvalueof{/agedness-columns/#2}} / \expandafter\thisrow{ext4_fs_size} * 1000000}{\thistable}
  %\addlegendentry{\pgfkeysvalueof{/fs-names/#1} \pgfkeysvalueof{/agedness-names/#2}}
}

% args are: hardware partition-size
\newcommandx{\startmsplot}[2]
{
   \begin{tikzpicture}
      \begin{axis}[
            width=\hsize,
            height=0.45\textheight,
      % scale only axis,
      % title=Insertion per second against Load Factor, 
      xlabel={Operations performed}, 
      ylabel={Grep cost (sec/GiB)}, 
      xmin=0,
      xmax=100, 
      ymin=0, 
      % ymax=50, 
      % xtick={0,20,40,60,80,100},
      % ytick={0,5,10,15,20,25,30,35,40,45,50},
      yticklabel style={rotate=90, anchor=south},
      grid=major, 
      scaled x ticks=false,
      scaled y ticks=false,
         legend entries = { \btrfs, \betrfs, \ext, \ftwofs, \xfs, \zfs, aged, unaged},
      legend columns=8,
      legend cell align=center,
      legend pos=north west,
      legend to name=mailserverplotslegend_#1_#2,
      ]
  	\addlegendimage{\pgfkeysvalueof{/fs-colors/btrfs}, mark=\pgfkeysvalueof{/fs-marks/btrfs}}
  	\addlegendimage{\pgfkeysvalueof{/fs-colors/betrfs}, mark=\pgfkeysvalueof{/fs-marks/betrfs}}
  	\addlegendimage{\pgfkeysvalueof{/fs-colors/ext4}, mark=\pgfkeysvalueof{/fs-marks/ext4}}
  	\addlegendimage{\pgfkeysvalueof{/fs-colors/f2fs}, mark=\pgfkeysvalueof{/fs-marks/f2fs}}
  	\addlegendimage{\pgfkeysvalueof{/fs-colors/xfs}, mark=\pgfkeysvalueof{/fs-marks/xfs}}
  	\addlegendimage{\pgfkeysvalueof{/fs-colors/zfs}, mark=\pgfkeysvalueof{/fs-marks/zfs}}
    %\pgfplotsinvokeforeach{btrfs, betrfs, ext4, f2fs, xfs, zfs} {
  	%  \addlegendimage{\pgfkeysvalueof{/fs-colors/#1}, mark=\pgfkeysvalueof{/fs-marks/#1}}
    %}
  	\addlegendimage{mark=\pgfkeysvalueof{/fs-marks/ext4}, color=\pgfkeysvalueof{/fs-colors/ext4},  style=\pgfkeysvalueof{/agedness-styles/aged}}
  	\addlegendimage{mark=\pgfkeysvalueof{/fs-marks/ext4}, color=\pgfkeysvalueof{/fs-colors/ext4}, style=\pgfkeysvalueof{/agedness-styles/clean}}
}

\NewEnviron{msplot}{\expandafter\startmsplot\BODY
\end{axis}
\end{tikzpicture}
}

\newcommand{\msplotsubcaption}[2]{\label{fig:ms:results:#1:#2} Results: \pgfkeysvalueof{/hardware-names/#1}, \pgfkeysvalueof{/partition-size-names/#2} partition}

% fs agedness hardware partition-size
\newcommand{\addmslayoutplot}[4]
{
  \pgfplotstableread{\mailserverDataFileName{#3}{#4}{#1}}\thistable
  \addplot[color=\pgfkeysvalueof{/fs-colors/#1}, style=\pgfkeysvalueof{/agedness-styles/#2}, line width=0.75pt, mark=\pgfkeysvalueof{/fs-marks/#1}, mark repeat=5] 
  table[x=\mailserverOperationCountColumn, y=\mailserverLayout{#2}] \thistable;
  %\addfsplot{#1}{#2}{\pullCountColumn}{\thisrow{\pgfkeysvalueof{/agedness-columns/#2}} / \expandafter\thisrow{ext4_fs_size} * 1000000}{\thistable}
  %\addlegendentry{\pgfkeysvalueof{/fs-names/#1} \pgfkeysvalueof{/agedness-names/#2}}
}

% args are: hardware partition-size
\newcommandx{\startmslayoutplot}[2]
{
	\begin{tikzpicture}
    \begin{axis}[
      width=\hsize,
      height=0.45\textheight,
      % scale only axis,
      % title=Insertion per second against Load Factor, 
      xlabel={Operations performed}, 
      ylabel={Dynamic layout score}, 
      xmin=0,
      xmax=100, 
      ymin=0, 
      % ymax=50, 
      % xtick={0,20,40,60,80,100},
      % ytick={0,5,10,15,20,25,30,35,40,45,50},
      yticklabel style={rotate=90, anchor=south},
      grid=major, 
      scaled x ticks=false,
      scaled y ticks=false,
      ]
}

\NewEnviron{mslayoutplot}{\expandafter\startmslayoutplot\BODY
\end{axis}
\end{tikzpicture}
}
%
% mailserver figures
%

\begin{figure}
  {\centering
   \ref{mailserverplotslegend_hdd_20gb}\\
    \begin{subfigure}{\columnwidth}
      \centering
	  \tikzsetnextfilename{mailserver-hdd}
      \begin{msplot}{hdd}{20gb}
        \addmailserverplot{betrfs}{clean}{hdd}{20gb}
        \addmailserverplot{betrfs}{aged}{hdd}{20gb}
        \addmailserverplot{btrfs}{clean}{hdd}{20gb}
        \addmailserverplot{btrfs}{aged}{hdd}{20gb}
        \addmailserverplot{ext4}{clean}{hdd}{20gb}
        \addmailserverplot{ext4}{aged}{hdd}{20gb}
        \addmailserverplot{f2fs}{clean}{hdd}{20gb}
        \addmailserverplot{f2fs}{aged}{hdd}{20gb}
        \addmailserverplot{xfs}{clean}{hdd}{20gb}
        \addmailserverplot{xfs}{aged}{hdd}{20gb}
        \addmailserverplot{zfs}{clean}{hdd}{20gb}
        \addmailserverplot{zfs}{aged}{hdd}{20gb}
      \end{msplot}
		\caption{Grep cost during mailserver workload (lower is better).}
    \end{subfigure}
    \begin{subfigure}{\columnwidth}
      \centering
	  \tikzsetnextfilename{mailserver-layout}
      \begin{mslayoutplot}{hdd}{20gb}
        \addmslayoutplot{betrfs}{clean}{hdd}{20gb}
        \addmslayoutplot{betrfs}{aged}{hdd}{20gb}
        \addmslayoutplot{btrfs}{clean}{hdd}{20gb}
        \addmslayoutplot{btrfs}{aged}{hdd}{20gb}
        \addmslayoutplot{ext4}{clean}{hdd}{20gb}
        \addmslayoutplot{ext4}{aged}{hdd}{20gb}
        \addmslayoutplot{f2fs}{clean}{hdd}{20gb}
        \addmslayoutplot{f2fs}{aged}{hdd}{20gb}
        \addmslayoutplot{xfs}{clean}{hdd}{20gb}
        \addmslayoutplot{xfs}{aged}{hdd}{20gb}
        \addmslayoutplot{zfs}{clean}{hdd}{20gb}
        \addmslayoutplot{zfs}{aged}{hdd}{20gb}
      \end{mslayoutplot}
		\caption{Mailserver layout (higher is better).}
    \end{subfigure}
    \caption{\label{fig:mailserver} Mailserver performance and layout scores.}
  }
\end{figure}
