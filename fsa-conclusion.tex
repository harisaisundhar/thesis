\section{Conclusion}\label{sec:fsa-conclusion}

The experiments above suggest that conventional wisdom on fragmentation, aging,
allocation and file systems is inadequate in several ways.

First, while it may seem intuitive to write data as few times as possible,
writing data only once creates a tension between the logical ordering
of the file system's current state and the potential to make modifications
without disrupting the future order. Rewriting data
multiple times allows the file system to maintain locality.  The overhead
of these multiple writes can be managed by rewriting data in batches, as is done in
write-optimized dictionaries.

% , and the overhead can
% be managed. The point is that not all writes are created equal; a logarithmic
% number of writes with very low write amplification will induce a lower write
% workload than a single write with large write amplification.

For example, in \betrfs, data might be written as many as a
logarithmic number of times, whereas in \ext, it will be written once,
yet \betrfs in general is able to perform as well as or better than an
unaged \ext file system and significantly outperforms aged \ext file
systems.

Second, today's file system heuristics are not able to maintain enough
locality to enable reads to be performed at the disks natural transfer
size.  And since the natural transfer size on a rotating disk is a
function of the seek time and bandwidth, it will tend to increase with
time. Thus we expect this problem to possibly become worse with newer
hardware, not better.

We experimentally confirmed our expectation that non-write-optimized
file systems would age, but we were surprised by how
quickly and dramatically aging impacts performance.
%It was somewhat surprising how quickly they all aged.  
This rapid aging is important: a user's
%(of at least certain workloads)
experience with unaged
file systems is likely so fleeting that they do not notice performance degradation.
Instead, the performance costs of aging
are built into their expectations of file system performance.

% dp: This para really doesn't do it for me
%%% There is quite a lot of future work to perform on aging.  Write
%%% optimization, and specifically \betrfs, appears to avoid aging, as
%%% predicted by our framework and design considerations.  Is there
%%% anything short of wholesale adoption of write optimization that could
%%% be applied to existing file systems to mitigate the effects of aging?
%%% Furthermore, SSDs present further issues, due to their performance
%%% profile and the FTL. Nonetheless, write optimization does generally
%%% improve aged performance on these devices, suggesting that integrating
%%% write-optimized data structures could further improve performance if
%%% they were incorporated to FTL designs.

Finally, because representative aging is a difficult goal, simulating multi-year workloads,
%Because of the perceived difficulty of aging file systems, 
many research 
papers benchmark on unaged file systems.
Our results indicate that it is relatively easy to quickly drive
a file system into an aged state---even if
this state is not precisely the state of the file system after, say, three
years of typical use---and this degraded state can be easily 
measured.

%%%  remark is that the current opinion generally appears to be
%%% that aging a file system is a difficult and elusive goal, since it may
%%% require simulating multi-year workloads. As a
%%% result, %of the perceived level of care required 
%%% many current research papers
%%% routinely benchmark on an unaged file system. 
%%% %We argue that it is
%%% %important to measure the expected state of use, even though precisely
%%% %recreating a specific aged state requires significant effort.  
%%% A key
%%% observation from our experiments, however, is that there is a
%%% precipitous drop from a clean, initial state, to an aged state in
%%% current file systems.  In other words, it is relatively easy to
%%% quickly drive a file system into a significantly aged state, even if
%%% this state is not precisely the state of the file system after three
%%% years of typical use.


%% Local Variables:
%% mode: latex
%% End:
