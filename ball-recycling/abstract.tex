% !TEX root =  paper.tex

\begin{abstract}

	Balls-and-bins games have been a successful tool for modeling load
	balancing problems.  In this paper, we study a new scenario, which we call
	the \defn{ball-recycling game}, defined as follows:
	\begin{displayquote}
		Throw $m$ balls into $n$ bins i.i.d.\ according to a given probability
		distribution $\p$.  Then, at each time step, pick a non-empty bin and
		\defn{recycle} its balls: take the balls from the selected bin and
		re-throw them according to $\p$.  
	\end{displayquote}
	This balls-and-bins game closely models memory-access heuristics in
	databases.  The goal is to have a bin-picking method that maximizes the
	\defn{recycling rate}, defined to be the expected number of balls recycled
	per step in the stationary distribution.

	We study two natural strategies for ball recycling: \FB, which greedily
	picks the bin with the maximum number of balls, and \RB, which picks a ball
	at random and recycles its bin.  We show that for general $\p$, \RB is
	$\Theta(1)$-optimal, whereas \FB can be pessimal. However, when $\p =
	\uni$, the uniform distribution, \FB is optimal to within an additive
	constant.

\end{abstract}

%%% Local Variables:
%%% mode: latex
%%% TeX-master: "paper.tex"
%%% End:

