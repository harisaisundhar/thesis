\section{Preemptive Splitting for \datastructs}
\label{sec:splitting}

Splits and merges pose problems for hand-over-hand locking in B-trees
(and \bets).  Hand-over-hand locking proceeds from root to leaf, but
splits and merges proceed from the leaves up.

An approach to solving this issue in B-trees is to use preemptive splitting and
merging~\cite{DBLP:journals/tos/Rodeh08}.  During a B-tree insert, if a child
already has the maximum number of children, then it is split while the
insertion thread still holds a lock on its parent.  Then the insertion can
release the parent's lock and proceed down the tree, assured that the child
will not need to split again as part of this insertion.  Analogously, deletions
merge a child with one of its neighbors if the child has the minimum number of
children.  This works because insertion and deletions can increase or decrease
the number of children of a node by at most 1.

This approach does not work in \bets, because a flush to a leaf could
cause that leaf to split multiple times.  In \datastruct with
flush-then-compact, we can move all pending messages along a
root-to-leaf path to the leaf before performing any compaction,
splits, or merges.  The total number of messages moved to the leaf is
bounded by $O(B\log_F N)$, i.e. the height of the tree times the
maximum amount of data that can be stored in branches at each trunk
node.  The leaf can therefore split into as many as $O(\log_F
N)$ new leaves of size $B$.  Similarly, a collection of flushes full
of delete messages to several leaves of a single parent can reduce the
parent's number of children by $O(\log_F N)$.

In practice, the height of the tree is less than 10 for typical
fanouts $F\approx 8$ and dataset size $N\leq 2^{80}$ key-value
pairs.

We extend preemptive splitting and merging to \datastructs as follows.
We reserve space in each node to accommodate up to $F + H$
children, where $H$ is an upper bound on the tree height, e.g. $H=10$.
We then apply preemptive splitting, except we preemptively split a
node during a flush if its fanout is above $F$.  For merges, we
take a similar approach.  If, during a flush, we encounter a node with
less than $F/2$ children, then we merge or rebalance it with one
of its siblings.

Thus all operations on the \datastruct---flushes, splits, and
merges---proceed from root to leaf and can therefore use
hand-over-hand locking.

The mechanisms for flush-then-compact make it easy to handle branches
during splits.  Recall that each branch can be marked dead or alive
for each child, and branches are refcounted and hence can be shared by
multiple trunk nodes. Thus we can split a trunk node by simply giving
its new sibling references to all the same branches as the node had
before the split.  In the new node, we copy the liveness information
for each branch along with the children that are moved to the new
sibling.

%% Therefore, preemptive
%% splitting is not enough to keep the number of pivots in the parent
%% below the maximum.  Restricting the flush rate to prevent this would
%% reduce update performance.

%% We solve this problem in \sysname's \bet by using \defn{postemptive} splitting.
%% Like preemptive splitting, this technique splits nodes as operations progress
%% down the tree with hand-over-hand locking. However, postemptive splitting only
%% splits nodes with \textit{more} pivots than the allowed maximum and is
%% permitted to temporarily exceed the maximum number of pivots in trunk index
%% nodes. Because fanout in a \bet is low, usually less than 20, there is more
%% than enough physical room in the node and then the split can be performed by
%% the next operation that touches the node.
